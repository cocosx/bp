\documentclass[a4wide,12pt]{report}
\usepackage[utf8]{inputenc}
\usepackage[IL2]{fontenc}
\usepackage{listings}
\usepackage{amssymb}
\usepackage{amsmath}
\usepackage{url}
\usepackage{graphicx}
\usepackage[czech]{babel}
\title{sds}
\author{Marek Bryša}
\date{Brno 2011}
\begin{document}


\chapter{Úvod}
\section{Historie firmy Oriflame}
Firma Oriflame byla založena v roce 1967 ve Švédsku bratry Jonasem a Robertem Jochnick. Cílem bylo nabídnout lidem přirozenou a přírodní péči o svou pleť. Místo aby nákladně budovali kamenné obchody, rozhodli se dostat prodej přímo do domovů a využít tak vrozenou podnikavost lidí. Tento základní koncept zůstáva po více než 40 let nezměněn. V roce 1990, po po pádu železné opony, firma expanduje po střední a východní Evropě a zakládá pobočku i v České republice. V roce 2001 dosahuje počet kosmetických poradců jednoho miliónu a obrat téměř 450 miliónů EUR.

Sortiment je opravdu široký, katalog obsahuje přes 1500 výrobků v cenách přibližně od 50 do 1000 Kč. Většina typů produktů obsahuje několik cenou a kvalitou odlišených řad. Celkově lze říci, že zákazník má možnost nákupu komplexní péče s téměř libovolným rozpočtovým omezením.
\section{Podmínky síťového prodeje}
V této části popíšeme fungování sítě Oriflame na českém trhu. V jiných zemích se mohou zejména parametry mírně lišit. Prodejcem (v terminologii Oriflame \emph{poradcem}) se zájemce stane vyplněním jednoduchého formuláře a zaplacením poplatku 99 Kč. Tím získá možnost výdělku následujícími způsoby:
\begin{enumerate}
\item Nákupem kosmetických výrobků v centrech Oriflame nebo na objednávkou přes internet za tzv. nákupní ceny, které jsou o 30\% nižší než prodejní (katalogové) ceny. Prodejce tak realizuje 30\% marži z prodeje sobě a svým zákazníkům (typicky rodině a známým).
\item Plnou hodnotu tzv. \emph{slevy z obratu} svého vlastního prodeje.
\item Budováním skupiny poradců, které k Oriflame přivedl, tzv. \emph{sponzoroval}. Pak mu náleží rozdíl mezi svou slevou z obratu a slevou z obratu lidí, které k Oriflame přivedl a jejich skupin.
\end{enumerate}

Rok je rovnoměrně rozdělen na 17 období, které odpovídají vydáním katalogů výrobků. Poradce, který podal v předchozích třech obdobích alespoň jednu objednávku, obdrží sadu tiskovin zdarma, jinak je pro něj téměř nutnost si tištěný katalog zakoupit za TODO:cena katalogu.

Firma dále nabízí pro začínající poradce ve čtyřech krocích motivaci ve formě věcných darů za splnění určitých objemů obratu, např. v prvním kroku za obrat 1250 Kč tašku, průvodce péčí o pleť a krém v celkové hodnotě 510 Kč.

Také jsou k dispozici úvěry do výše 7000 Kč na zaplacení za zobží, které hodlá poradce dodat zákazníkum a nemusí tak od nich vybírat peníze předem.

Oriflame poskytuje garanci vrácení peněz do 30 dnů od nákupu výrobku bez udání důvodu a to i v případě, že výrobek obsahuje minimálně 80\% původního objemu.
\subsection{Skupiny a slevy z obratu}
Každému výrobku jsou v ceníku přiděleny čtyři hodnoty:
\begin{itemize}
\item PC - doporučená prodejní cena spotřebitelům.
\item NC - nákupní cena včetně DPH. Za tu poradci mohou zobží zakoupit v centrech Oriflame.
\item OO - obchodní obrat. Typicky je roven nákupní ceně bez DPH. V případě nespotřebních produktů (např. kartáč na vlasy, houba na mytí) je ještě přibližně poloviční.
\item BO - bodové ohodnocení. Na inflaci nezávislý počet bodů, které poradce nákupem zboží získá. V současnoti odpovídá přibližně 13,80 Kč obchodního obratu.
\end{itemize}
Výjimku tvoří např. tiskoviny, vzorky a oblečení s logem Oriflame, které mají pouze nákupní cenu,  nejsou určeny k dalšímu prodeji a slouží jen jako pomůcka pro poradce.

Poradce dále získá body všech lidí které přivedl a těch pod nimi. Na konci každého katalogového období dochází k vyhodnocení. Sečtou se všechny body a podle následujícího klíče se určí procentní úroveň:
\begin{center}
\begin{tabular}{|l|c|c|c|c|c|c|c|}
\hline
body & $\geq$200 & $\geq$600 & $\geq$1200 & $\geq$2400 & $\geq$4000 & $\geq$6600& $\geq$10000\\\hline
úroveň & 3\% & 6\% & 9\% & 12\% & 15\% & 18\% & 21\%\\\hline
\end{tabular}
\end{center}
Toto se provede i pro všechny podskupiny. Poradce pak získá za každou podskupinu: obchodní obrat skupiny $\cdot$ (procentní úroveň poradce $-$ procentní úroveň skupiny). Tím je zajištěno, že člověk, který se stal poradcem Oriflame později, ale je schopnější, může dosáhnout vyššího výdělku než ten, kdo jej přivedl. Poradce také obdrží přímo podle své procetní úrovně slevu ze svého vlastního prodeje.

Všechny tyto peníze poradce dostane na účet. Podmínkou je, že jeho vlastní prodej musí dosáhnout 100 bodů. To je zdůvodněno tím, že poradce musí být sám schopným prodejcem, aby mohl učit ostatní.

Při dosažení 12\% úrovně se poradce stává tzv. manažerem, pří 21\% direktorem. Za to má možnost získat věcné i finanční prémie, účast na školeních aj.
\chapter{Popis modelu}
\chapter{Experimenty}
\begin{thebibliography}{9}

\bibitem{fd}

\end{thebibliography}
\url{http://www.oriflame.com/About_Oriflame/History/}
cenik 4/2011
manual kosmetickeho poradce
\end{document}

